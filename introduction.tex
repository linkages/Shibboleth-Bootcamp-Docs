% setup
\documentclass[hyperref={colorlinks=true},professionalfonts]{beamer}
% \usepackage[latin1]{inputenc}
\usepackage{listings}
\usepackage{hyperref}

% any shorthands
\newcommand{\us}[1]{ \_#1\_ } % we do so many underscored identifiers
\newcommand{\email}[1]{\href{mailto:#1}{#1}}

% presentation configuration
\usetheme{Warsaw}
\usecolortheme{crane}

\title[Shibboleth Bootcamp]{An Introduction to Shibboleth}
\institute{University of Florida}
\author{UF IT/CNS/Open Systems Group}

\date{\today}

% document
\begin{document}

\begin{frame}
\titlepage

\center{
Eli Ben-Shoshan (\email{ebs@ufl.edu}) \\ 
Martin Smith (\email{smithmb@ufl.edu}) \\
Laura Guazzelli (\email{laura2@ufl.edu}) \\
}
\end{frame}


\begin{frame}
\frametitle{Important references}
\begin{itemize}
  \item UF IT - Shibboleth \\ \url{http://www.it.ufl.edu/identity/shibboleth}
  \item CNS/Open Systems Group - Shibboleth \\ \url{http://open-systems.ufl.edu/shibboleth}
  \item Internet2 - Shibboleth \\ \url{https://spaces.internet2.edu/display/SHIB2/Home}
\end{itemize}
\end{frame}

\begin{frame}
\frametitle{Goals}

What you should know by the end:
\begin{itemize}
\item How to install SP software
\item General understanding about Shibboleth
\item How to configure SP software
\end{itemize}

\bigskip

What you should have done by the end
\begin{itemize}
\item Installed your SP
\item Learned how to protect your content
\end{itemize}

\end{frame}

\begin{frame}
\frametitle{Requirements}

You should have the following ready for this class:
\begin{itemize}
\item A test/dev machine at your office
\item Access to your test/dev machine
\item Capability to install software on test/dev machine
\item Willingness to have your test/dev machine go down for a bit
\end{itemize}

\end{frame}

\begin{frame}
\frametitle{Definitions}
\begin{itemize} 
\item Shibboleth Service Provider (SP) \\ You and the SP software that you install and maintain on your webserver.
\item Shibboleth Identity Provider ( IdP ) \\ The central authentication server. The IdP authenticates the user and vends attributes about the user.
\end{itemize}
\end{frame}

\begin{frame}
\frametitle{Definitions (continued)}
\begin{itemize}
\item Security Assertion Markup Language (SAML) \\ An XML standard for exchanging authentication and authorization data.
\item Service Endpoint \\ A set of URLs on the SP and IdP that are used to transfer SAML documents.
\item Metadata \\ A document that names all of the service endpoints.
\end{itemize}
\end{frame}

\begin{frame}
\frametitle{Definitions (continued)}
\begin{itemize}
\item Entity Identifier (entityID) \\ A universal resource name (URN) that identifies your SP
\item All entityID's for UF take the following form: \begin{itemize}
\item \texttt{urn:edu:ufl:prod:XXXXX} for production
\item \texttt{urn:edu:ufl:test:XXXXX} for test
\item \texttt{urn:edu:ufl:dev:XXXXX} for development
\end{itemize}
\end{itemize}
\end{frame}

\begin{frame}
\frametitle{Shibboleth software on your SP}
The Shibboleth software that runs on your SP is setup as follows:
\begin{itemize}
\item \textbf{Shibboleth module} that runs in your webserver (IIS/Apache) that maps URIs to requests and talks to Shibboleth daemon
\item \textbf{Shibboleth daemon} that does all the heavy lifting, decrypts SAML, extracts attributes
\end{itemize}
\end{frame}

\begin{frame}
\frametitle{Software Install}
Official directions are here: \\
{\small \url{http://www.it.ufl.edu/identity/shibboleth/technical.html}}

\bigskip
The directions are similar between Windows/IIS and Unix/Apache.
\end{frame}

\begin{frame}
\frametitle{Install the software - Windows}
{\tiny See \url{http://www.it.ufl.edu/identity/shibboleth/technicalIIS.html}.}

\begin{itemize}
\item Download the latest MSI installer from this page for your platform and install it, then reboot
\item Please do not change any defaults offered by the installer unless absolutely necessary
\item Verify that the installer correctly created an ISAPI filter on your site and configured the Shibboleth daemon as a Windows service
\end{itemize}

\end{frame}

\begin{frame}
\frametitle{Install the software – RHEL}
{\tiny See \url{http://www.it.ufl.edu/identity/shibboleth/technicalapache.html}.}

\begin{itemize}
\item Download and install the RPMs from this page for your platform
\item Edit Apache config to load the shibboleth module and set UseCanonicalName
\item Restart Apache and start the Shibboleth daemon
\end{itemize}

\end{frame}

\begin{frame}[fragile]
\frametitle{Configuring Shibboleth Daemon}
All configuration for daemon is in the \texttt{shibboleth2.xml} file. Get the template from the Open Systems site:
\url{http://open-systems.ufl.edu/shibboleth}
\bigskip

Place the file in the correct location: \\ \bigskip
\textbf{Windows} - \begin{verbatim} C:\opt\shibbolethsp\etc\shibboleth\shibboleth2.xml \end{verbatim}
\textbf{Unix} - \begin{verbatim} /etc/shibboleth/shibboleth2.xml \end{verbatim}
\end{frame}

\begin{frame}
\frametitle{Configuring Shibboleth Daemon (continued)}
\textbf{Update shibboleth2.xml} template, replacing variables: \\
\begin{itemize}
\item \us{HOSTNAME} - fully qualified domain of your site 
\item \us{URN} - entityID assigned to you by Bridges IAM Admin
\end{itemize}
For Windows you also have
\begin{itemize}
\item \us{SITEID} - IIS "Site Identifier" for this website
\end{itemize}
\end{frame}

\begin{frame}[fragile]
\frametitle{Configuring Shibboleth Daemon (continued)}
\textbf{Remove} the sp-cert.pem and sp-key.pem from the Shibboleth configuration directory for your platform \\ \bigskip
\textbf{Windows} - \begin{verbatim} C:\opt\shibbolethsp\etc\shibboleth \end{verbatim}
\textbf{Unix} - \begin{verbatim} /etc/shibboleth \end{verbatim}
\end{frame}

\begin{frame}[fragile]
\frametitle{Configure Shibboleth Daemon (continued)}
\textbf{Generate} the key and certificate: \\ \bigskip
\textbf{Windows} - \texttt{keygen.bat -h \us{HOSTNAME} -e \us{URN}} \\
\bigskip
\textbf{Unix} - \texttt{keygen.sh -h \us{HOSTNAME} -e \us{URN}} \\
\end{frame}

\begin{frame}
\frametitle{Configure Shibboleth Daemon}
\textbf{Rename} the generated files: \\ \bigskip
\texttt{sp-cert.pem} should be renamed to \texttt{\us{HOSTNAME}.cert} \\
\bigskip
\texttt{sp-key.pem} should be renamed to \texttt{\us{HOSTNAME}.key} \\
\bigskip
Now, \textbf{restart} the shibboleth daemon.
\end{frame}

\begin{frame}
\frametitle{Checking your install}
\textbf{If all went well}, then you should have a shibboleth daemon running 
and the webserver should respond with your SP's metadata at this URL: \\ \bigskip
\textbf{\texttt{http://\us{HOSTNAME}/Shibboleth.sso/Metadata}}
\end{frame}

\begin{frame}
\frametitle{Check your install}
\textbf{Review} your metadata:
\begin{itemize}
\item Make sure the \textbf{entityID is correct} for this SP
\item Make sure there is \textbf{at least one} of these services defined:
\begin{itemize}
\item AssertionConsumerService
\item ManageNameIDService
\item SingleLogoutService
\end{itemize}
\end{itemize}
\end{frame}

\begin{frame}
\frametitle{Service provider completed}
{\large \textbf{Congratulations!} Your SP is now configured.} \\
\bigskip
\textbf{Submit your Metadata} for inclusion in the IdP using \url{https://open-systems.ufl.edu/shibmeta}.

\bigskip \bigskip
\textbf{Until this happens} your will get an error message on your SP: \\ \bigskip
\textit{Error Message: SAML 2 SSO profile is not configured for relying party urn:edu:ufl:XXXX:YYYYY}
\end{frame}

\begin{frame}
\frametitle{Protecting Content}
Two ways to accomplish content protection:
\begin{itemize}
\item Modify shibboleth2.xml
\item Modify .htaccess (Apache only)
\end{itemize}
\end{frame}

\begin{frame}
\frametitle{Protecting Content (shibboleth2.xml)}
This can be used for both IIS and Apache, but this is \textbf{the only way to protect content in IIS}.
\begin{itemize}
\item Add a \texttt{Path} element to the \texttt{Host} element 
\item Add a \texttt{AccessControl} element to \texttt{Path} element
\item Add a \texttt{Rule} element to the \texttt{AccessControl} element
\end{itemize}
\end{frame}

\begin{frame}[fragile]
\frametitle{Protecting Content, Simple (shibboleth2.xml)}
\begin{lstlisting}[language=XML,basicstyle=\ttfamily \small]
<RequestMapper>
<RequestMap>
<Host name="example.com">
<Path name="secure" 
   requireSession="true" authType="shibboleth">
<AccessControl>
<Rule require="primary-affiliation">S</Rule>
</AccessControl>
</Path>
</Host>
</RequestMap>
</RequestMapper>
\end{lstlisting}

\end{frame}

\begin{frame}[fragile]
\frametitle{Protecting Content, Complex (shibboleth2.xml)}
\begin{lstlisting}[language=XML,basicstyle=\ttfamily \small]
<RequestMapper>
<RequestMap>
<Host name="example.com" 
   requireSession="true" authType="shibboleth">
<Path name="secure">
<AccessControl>
<OR>
<Rule require="primary-affiliation">S</Rule>
<Rule require="primary-affiliation">F</Rule>
</OR>
</AccessControl>
</Path>
</Host>
</RequestMap>
</RequestMapper>
\end{lstlisting}

\end{frame}

\begin{frame}
\frametitle{Protecting Content (.htaccess)}
Much easier to use and maintain. \\ \bigskip
If you are using Apache, use this method.
\end{frame}

\begin{frame}[fragile]
\frametitle{Protecting Content (.htaccess)}
\center{Simple Example} \\
\begin{lstlisting}[language=XML,basicstyle=\ttfamily \small]
AuthType Shibboleth
ShibRequireSession On
Require valid-user
\end{lstlisting}
\end{frame}

\begin{frame}[fragile]
\frametitle{Protecting Content (.htaccess)}
\center{Complex Example} \\
\begin{lstlisting}[language=XML,basicstyle=\ttfamily \small]
AuthType Shibboleth
ShibRequireSession On
Require primary-affliation ~ S|F
\end{lstlisting}
\end{frame}

\begin{frame}
\frametitle{Questions?}
\center{Thank you.}
\end{frame}

\end{document}
